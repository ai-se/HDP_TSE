\section{Motivation}
\label{sec:Motivation}

One reason to explore transfer learning
is to study the nature of
generality in software engineering.  Professional
societies assume such generalities exist when they
offer lists of supposedly general ``best practices'':
\bi
\item For example, the  IEEE 1012 standard for software
  verification~\cite{1012};
\item
  Also, Endres & Rombach offer
dozens of lessons of software
engineering~\cite{endres03};
\item Further, many other
widely-cited researchers dothe same; e.g.
Glass~\cite{glass02}; Jones~\cite{jones10};
Boehm~\cite{hoehm00b}.
\item
  More generally, Budgen &Kitchenham seek to
reorganize SE research using general conclusions
drawn from a larger number of
studies~\cite{budgen06,budgen09}.
\ei
However,
given the constant pace of change within SE, can we trust
those supposed generalities? 
Numerous {\em local learning} results show that we
should mistrust general conclusions (made over a
wide population of projects) since they may not hold
for projects.  Posnett et al.~\cite{posnett11}
discuss {\em ecological inference} in software
engineering, which is the concept that what holds
for the entire population also holds for each
individual.  They learn models at different levels
of aggregation (modules, packages, files) and show
that models that work at one level of aggregation
can be sub-optimal at others.  For example, Yang et
al.~\cite{yang11}, Bettenburg et
al.~\cite{betten14}, and Menzies et al.~\cite{me12d}
all explore the generation of models using {\em all}
data versus {\em local} samples that more specific
to particular test cases. These papers report that
better models (sometimes with much lower variance in
their predictions) are generated from local
information.
These results have an unsettling effect on anyone
struggling to propose policies for an organization.
If all prior conclusions can change for the new
project, or some small part of a project, how can
any manager ever hope to propose, and defend, IT
policies (e.g. when should some module be inspected,
when should it be refactored, where to focus
expensive testing procedures, etc)?

If we cannot {\em generalize} to all projects and all parts
of current projects, perhaps a more achievable goal is
the {\em stabilize} the pace of conclusion change. 
While it may be 
a fool's errand  and wait for  eternal and global SE
conclusions, one possible approach is for organizations
to declare $N$ prior projects as {\em reference projects},
from which lessons learned will be transferred to new projects.
In practice, using such reference sets requires three processes:
\be
\item Finding the reference sets (this paper shows that finding
  them may not be too complex a task, at least for defect prediction).
  \item Recognizing when to update  the reference set. In practice,
  this could be as simple as noting when predictions start failing for new projects-- at which time, we would loop to point \#1.
\item Transferring
  lessons from the reference set to new projects.  
  \ee
In this approach, the policies of the organization will be
stable just as long as the reference set is not updated.
In this paper, we do not address the pace of change in the reference set
(that is left for future work).
Rather, we focus on point \#3: transferring lessons from
the reference set to new projects. To support this third point,
we need to  resolving the problems
  that this paper addresses (data expressed in different terminology,
  cannot transfer till there is enough data to match old projects to new).

  

