\section{Related Work}
\subsection{Related Work on Transfer Learning}
In the machine learning literature, the 2010 article by
Pan and Yang~\cite{Pan10} is the definitive definition of
transfer learning.

Pan and Yang state that transfer learning is defined over a {\em domain} $D$,
which is composed of pairs of
examples $X$ and a probability distribution about those examples $P(X)$;
i.e., \mbox{$D = \{ X, P(X) \}$}.
This $P$ distribution represents what class values to expect, given the $X$ values.


The transfer learning {\em task} $T$ is to learn a function $f$ that predicts labels $Y$; i.e.,
$T = \{ Y, f \}$.
Given a new example $x$, the intent is that the function can produce a correct
label $y\in Y$; i.e., $y = f(x)$ and $x\in X$.
According to Pan and Yang, synonyms for transfer learning include, learning
to learn, life-long learning, knowledge transfer, inductive transfer,
multitask learning, knowledge consolidation, context-sensitive
learning, knowledge-based inductive bias, metalearning, and
incremental/cumulative learning.

Pan and Yang~\cite{Pan10} define four types of transfer learning:
\squishlist
\item
When moving from some source domain    to the target domain,   {\em instance-transfer} methods
provide example data for model building in the target;
\item
{\em Feature-representation transfer} synthesizes example data for
model building;
\item
{\em Parameter transfer} provides parameter terms for existing models;
\item
and {\em Relational-transfer} provides mappings between term parameters.
\squishend
From a business perspective, we can offer the following examples of
how to use these four kinds of transfer. Take the case where a company
is moving from Java-based desktop application development to
Python-based web application development. The project manager for the
first Python webapp wants to build a model that helps her predict
which classes have the most defects so that she can focus on system
testing:

\squishlist
\item
{\em Instance-transfer} tells her which Java project data are relevant
for building her Python defect prediction model.
\item
{\em Feature-representation transfer} will create synthesized Python
project data based on analysis of the Java project data that she can
use to build her defect prediction model.
\item
If defect prediction models previously existed for the Java projects,
{\em parameter transfer} will tell her how to weight the terms in old
models to make those model are relevant for the Python projects.
\item
Finally, {\em relational-transfer} will tell her how to translate some
JAVA-specific concepts (such as metrics collected from JAVA interfaces
classes) into synonymous terms for Python (note that this last kind
of transfer is very difficult and, in the case of SE, the least
explored).
\squishend
In the SE literature, methods for CPDP using same/common metrics sets are examples
of {\em instance transfer}. As to the other kinds of transfer, there is some work in the effort estimation literature of using
genetic algorithms to automatically learn weights for different parameters~\cite{sigweni2014feature}. Such work is an example of {\em parameter transfer}.
To the best of our knowledge, there is no work on feature-representation transfer, but research into automatically
learning APIs between programs~\cite{dallmeier2012automatically} might be considered a close analog.

In the survey of Pan and Yang~\cite{Pan10}, most transfer learning algorithms in these four types of transfer learning assume the same feature space.
In other words, the surveyed transfer learning studies in~\cite{Pan10} focused on different distributions between source and target `domains or tasks'
under the assumption that the feature spaces between source and target domains are same. However, Pan and Yang discussed the need for transfer learning between source and target that have different feature spaces and referred to this kind of transfer learning as {\em heterogeneous transfer learning}.

A recent survey of transfer learning by Weiss et al. published in 2016~\cite{Weiss2016} categorizes transfer learning approaches in homogeneous or heterogeneous transfer learning based on the same or different feature spaces respectively. Weiss et al. put the four types of transfer learning by Pan and Yang into homogeneous transfer learning~\cite{Weiss2016}. For heterogeneous transfer learning, Weiss et al. divide related studies into two sub-categories: {\em symmetric transformation} and {\em asymmetric transformation}~\cite{Weiss2016}. Symmetric transformation finds a common latent space whether both source and target can have similar distributions while Asymmetric transformation aligns source and target features to form the same feature spaces~\cite{Weiss2016}.

By the definition of Weiss et al., HDP is an example of heterogeneous transfer learning based on asymmetric transformation to solve issues of CPDP using heterogeneous metric sets. We discuss the related work about CPDP based on transfer learning concept in the following subsection.

\subsection{Related Work on Defect Prediction}
\label{sec:Background}
%\subsection{Cross-project Defect Prediction}
Recall from the above that we distinguish cross-project defect prediction (CPDP) from within-project defect prediction (WPDP).
The CPDP approaches have been studied by many researchers
of late~\cite{Canfora13,Ma12,Nam13,Panichella14,Rahman12,Ryu14,Ryu15,Turhan09,Zhang15,Zimmermann09}. Since the performance
of CPDP is usually very poor~\cite{Zimmermann09}, researchers have proposed various techniques to
improve CPDP~\cite{Canfora13,Ma12,Nam13,Panichella14,Ryu14,Ryu15,Turhan09,Watanabe08}. In this section, we discuss CPDP studies in terms of metric sets in defect prediction datasets.

% In the previous studies, cross-prediction was
% applied on datasets with the same metric set or researchers simply used
% following ways to overcome the limitation about different metric sets:
% % ~\sung{I think you need to organize previous work based on this.
% % Say, A did something. B did something. However, they applied on datasets with the same metric sets.
% % C did something. D did something. Howerver, they used common metric among existing project datasets.
% % \ldots
% % }
% \squishlist
% 	\item Collecting the same metric set for projects used in an empirical study.
% 	\item Using common metrics among existing project datasets.
% 	\item Conducting cross-prediction within a project group consisting of
% 	existing projects having the same metric set.
% \squishend

\subsubsection{CPDP Using Same/common Metric Sets}
Watanabe et al. proposed the metric compensation approach for
CPDP~\cite{Watanabe08}. The metric compensation transforms a target dataset
similar to a source dataset by using the average metric
values~\cite{Watanabe08}.
To evaluate the performance of the metric compensation, Watanabe et al. collected
two defect datasets with the same metric set (8 object-oriented
metrics) from two software projects and then conducted
CPDP~\cite{Watanabe08}.

Rahman et al. evaluated the CPDP performance in terms of
cost-effectiveness and confirmed that the prediction performance of CPDP is
comparable to WPDP~\cite{Rahman12}. For the empirical study, Rahman et al.
collected 9 datasets with the same process metric set~\cite{Rahman12}.

Fukushima et al. conducted an empirical study of just-in-time defect prediction
in the CPDP setting~\cite{Fukushima14}. They used 16
datasets with the same metric set~\cite{Fukushima14}. The 11 datasets were
provided by Kamei et al. but 5 projects were newly collected with the same metric
set used in the 11 datasets~\cite{Fukushima14,Kamei13}.

However, collecting datasets with the
same metric set might limit CPDP. For example, if
existing defect datasets contain object-oriented metrics such as CK
metrics~\cite{Basili96}, collecting the same object-oriented metrics is
impossible for projects that are written in non-object-oriented languages.

Turhan et al. proposed the nearest-neighbour (NN) filter to improve the
performance of CPDP~\cite{Turhan09}. The basic idea of the NN
filter is that prediction models are built by source instances that are
nearest-neighbours of target instances~\cite{Turhan09}. To conduct
CPDP, Turhan et al. used 10 NASA and SOFTLAB datasets in the
PROMISE repository~\cite{promise12,Turhan09}.
%With the NN filter, cross-prediction results improved from 0.26 to 0.35 in
%terms of average f-measure. However, within-prediction results were still the
% best (0.37)~\cite{Turhan09}.

Ma et al. proposed Transfer Naive Bayes (TNB)~\cite{Ma12}. The TNB builds a
prediction model by weighting source instances similar to target
instances~\cite{Ma12}. Using the same datasets used by Turhan et al., Ma et al.
evaluated the TNB models for CPDP~\cite{Ma12,Turhan09}.
% The cross-prediction performance of TNB
% (0.39) outperformed that of the NN filter (0.35) in terms of average f-measure~\cite{Ma12}.

Since the datasets used in the empirical studies of Turhan et al. and Ma et al.
have heterogeneous metric sets, they conducted CPDP using the common
metrics~\cite{Ma12,Turhan09}. There is another CPDP study with the top-K
common metric subset~\cite{He14subset}. However, as explained in
Section~\ref{sec:Introduction}, CPDP using common metrics is
worse than WPDP~\cite{He14subset,Turhan09}.
%   The NASA and SOFTLAB datasets have different
% metric sets (21 to 39 metrics)~\cite{Turhan09}. Among them, only 17 common
% metrics were used in their empirical studies.



Nam et al. adapted a state-of-the-art
transfer learning technique called Transfer Component Analysis (TCA) and
proposed TCA+~\cite{Nam13}. They used 8
datasets in two groups, ReLink and AEEEM, with 26 and 61 metrics
respectively~\cite{Nam13}.
%
% He et al. conducted the empirical study for defect prediction
% using a simplified metric set~\cite{He14subset}. They conducted CPDP on 34
% datasets with the same metric set (20 code metrics) and showed CPDP

However, Nam et al. could not conduct CPDP between ReLink and AEEEM because they
have heterogeneous metric sets.
Since the project pool with the same metric set is very limited, conducting
CPDP using a project group with the same metric set can be
limited as well. For example, at most 18\% of defect datasets in the
PROMISE repository have the same metric set~\cite{promise12}. In other words,
we cannot directly conduct CPDP for the 18\% of the defect datasets by
using the remaining (82\%) datasets in the PROMISE
repository~\cite{promise12}.


There are other CPDP studies using datasets with the same metric sets or using common metric sets~\cite{Canfora13,me12d,promise12,Panichella14,Zhang15,Ryu14,Ryu15}.
%Panichella. Canfora Ryu, Zhang(comsac)
Menzies et al. proposed a local prediction model based on clustering~\cite{me12d}. They used seven defect datasets with 20 object-oriented metrics from the PROMISE repository~\cite{me12d,promise12}.
Canfora et al., Panichella et al., and Zhang et al.
used 10 Java projects only with the same metric set from the PROMISE
repository~\cite{Canfora13,promise12,Panichella14,Zhang15}.
Ryu et al. proposed the value-cognitive boosting and transfer cost-sensitive boosting approaches for CPDP~\cite{Ryu14,Ryu15}. Ryu et al. used common metrics in NASA and SOFTLAB datasets~\cite{Ryu14} or Jureczko datasets with the same metric set from the PROMISE repository~\cite{Ryu15}. These recent studies for CPDP did not discuss about the heterogeneity of metrics across project datasets.
% ~\sung{say how many \% of PROMISE projects have the same dataset.}
% although there are lots of existing datasets available.

% Cross-prediction with TCA+ showed comparable results to within-prediction in average f-measure on 8 open source projects~\cite{Nam13}.

% Fukushima et al. conducted an empirical study of just-in-time defect prediction
% under cross-project settings on 11 open source projects with the same metric set~\cite{Fukushima14}.
% They observed cross-prediction models can be
% improved in terms of AUC by selecting a source dataset similar to a target
% dataset~\cite{Fukushima14}.


Zhang et al. proposed the universal model for CPDP~\cite{Zhang14}.
%Since the individual project may have its specific
%defect characteristic, the universal model may not work for all
%projects~\cite{Zhang14}. To resolve this limitation, Zhang et al. proposed
%context-aware rank transformations that change metric values ranged from 1 to
% 10 across all projects~\cite{Zhang14}. In this way,
The universal model is
built using 1398 projects from SourceForge and Google code and leads to
comparable prediction results to WPDP in their experimental
setting~\cite{Zhang14}.

However, the universal defect prediction model may be difficult to apply for the
projects with heterogeneous metric sets since the universal
model uses 26 metrics including code metrics, object-oriented metrics, and
process metrics. In other words, the model can only be applicable for
target datasets with the same 26 metrics. In the case where the target project
has not been developed in object-oriented languages, a universal model built
using object-oriented metrics cannot be used for the target dataset.
%  or need to costly build a new universal model with the same
% metric set of the target dataset.

\subsubsection{CPDP Using Heterogeneous Metric Sets}
He et al.~\cite{He14} addressed the limitations due to
heterogeneous metric sets in CPDP studies listed
above.
Their approach, CPDP-IFS, used distribution
characteristic vectors of an instance as metrics.
The prediction performance of their best approach is comparable to or
helpful in improving regular CPDP models~\cite{He14}.

However, the approach by He et al. is not compared with WPDP~\cite{He14}.
Although their best approach is helpful to improve regular CPDP models, the
evaluation might be weak since the prediction performance of a regular CPDP is
usually very poor~\cite{Zimmermann09}. In addition, He et al. conducted
experiments on only 11 projects in 3 dataset groups~\cite{He14}.

Jing et al. proposed heterogeneous cross-company defect prediction based on the extended canonical correlation analysis (CCA+)~\cite{Jing15} to address the limitations of heterogeneous metric sets. Their approach adds dummy metrics with zero values for non-existing metrics in source or target datasets and then transforms both source and target datasets to make their distributions similar. CCA+ was evaluated on 14 projects in four dataset groups.
%One of issues of CCA+ is that it was not properly evaluated on cross-company dataset groups. For example, ReLink datasets are not from the same company but they just have same metric sets.

% In our previous study, we used two groups of datasets, ReLink and
% AEEEM~\cite{Nam13}.
We propose HDP to address the above
limitations caused by projects with heterogeneous metric sets.
Contrary to the study by He et al.~\cite{He14}, we compare HDP to
WPDP, and HDP achieved better or comparable prediction performance to WPDP in
about 71\% of predictions.
Comparing to the experiments for CCA+~\cite{Jing15} with 14 projects, we conducted more extensive experiments with 34 projects in 5 dataset groups. In addition, CCA+ transforms original source and target datasets so that it is difficult to directly explain the meaning of metric values generated by CCA+~\cite{Jing15}. However, HDP keeps the original metrics and builds models with the small subset of selected and matched metrics between source and target datasets in that it can make prediction models simpler and easier to explain~\cite{Nam15HDP,Shihab14}. In
Section~\ref{sec:Approach}, we describe our approach in detail.

% In this section, we explain transfer learning on different domains.
% Transfer learning is one of the recent very active research topics in the
% machine learning community~\cite{Pan10}. Transfer learning can reduce expensive
% costs incurred while collecting and labeling training data by transferring
% knowledge from existing data and models~\cite{Pan10}.
%
% To define transfer learning formally, Pan et al. gave the definition of a
% domain; ``a domain, $D$, consists of two components: a feature space $\chi$ and
% a marginal distribution $P(X)$, where $X=\{x_1,...,x_n\}\in
% \chi$''~\cite{Pan10}. According to this definition, a domain can be
% represented as follows: $D=\{\chi,P(X)\}$. When we say two domains are different, it
% represents different feature spaces, $\chi$, or different distributions, $P(X)$.
%
% Most transfer learning studies focused on different distributions by
% assuming feature spaces are the same across datasets, but less focused on
% different feature spaces~\cite{Pan10}.
% Since our study aims to resolve the issue on defect prediction on
% datasets with different feature spaces, the term `different
% domains' stands for `different feature spaces of datasets' throughout the paper.
%
%
% Dai et al.~\cite{Dai08} proposed Translated Learning to address transfer
% learning on different feature spaces.
% The core approach used in translated learning is to identify co-occurrence features, which are
% linked between source and target features to migrate knowledge of source data to
% models for target data~\cite{Dai08}. Yang et al.~\cite{Yang09} introduced
% heterogeneous transfer learning for image clustering by using text
% data, which have a different feature space to image data. Zhu et
% al.~\cite{Zhu11} proposed heterogeneous transfer learning for image
% classification using text data that has different feature spaces. However, all
% these studies utilize tags given to images on web sites to link/bridge difference feature sets so that it is
% difficult to directly apply their approaches for our cross-domain defect
% prediction.
% % Translated learning is an independent
% % algorithm from existing machine learning algorithms such as Decision tree,
% % Random Forest, Logistic Regression and etc., which are widely used for defect prediction.
%
% Inspired by translated and heterogeneous transfer learning, we design
% our approaches to identify co-occurrence features between source and target
% datasets for the cross-domain defect prediction. Our approaches are based on
% statistical information and existing approaches that might be used to measure
% similarity between two features.
% %working together with
% %existing machine learning algorithms for defect prediction.
% In Section~\ref{sec:analyzers}, we describe in detail how to identify
% co-occurrence features for our cross-domain defect prediction.


% \subsubsection{Random Analyzer}
% We defined random analyzer, where we match srouce and target feautres randomly
% to evaluate if the proposed analyzers work better than the random analyzer.
