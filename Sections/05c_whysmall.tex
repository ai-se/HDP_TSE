

\begin{figure*}[t!]
	\renewcommand{\baselinestretch}{0.8}\begin{center}
		{\scriptsize
			\begin{tabular}{c|l|p{4.7in}}
				amc & average method complexity & e.g. number of JAVA byte codes\\\hline
				avg\, cc & average McCabe & average McCabe's cyclomatic complexity seen
				in class\\\hline
				ca & afferent couplings & how many other classes use the specific
				class. \\\hline
				class. \\\hline
				cam & cohesion amongst classes & summation of number of different
				types of method parameters in every method divided by a multiplication
				of number of different method parameter types in whole class and
				number of methods. \\\hline
				cbm &coupling between methods &  total number of new/redefined methods
				to which all the inherited methods are coupled\\\hline
				cbo & coupling between objects & increased when the methods of one
				class access services of another.\\\hline
				ce & efferent couplings & how many other classes is used by the
				specific class. \\\hline
				dam & data access & ratio of the number of private (protected)
				attributes to the total number of attributes\\\hline
				dit & depth of inheritance tree &\\\hline
				ic & inheritance coupling &  number of parent classes to which a given
				class is coupled (includes counts of methods and variables inherited)
				\\\hline
				lcom & lack of cohesion in methods &number of pairs of methods that do
				not share a reference to an case variable.\\\hline
				locm3 & another lack of cohesion measure & if $m,a$ are  the number of
				$methods,attributes$
				in a class number and $\mu(a)$  is the number of methods accessing an
				attribute, 
				then
				$lcom3=((\frac{1}{a} \sum, j^a \mu(a, j)) - m)/ (1-m)$.
				\\\hline
				loc & lines of code &\\\hline
				max\, cc & maximum McCabe & maximum McCabe's cyclomatic complexity seen
				in class\\\hline
				mfa & functional abstraction & number of methods inherited by a class
				plus number of methods accessible by member methods of the
				class\\\hline
				moa &  aggregation &  count of the number of data declarations (class
				fields) whose types are user defined classes\\\hline
				noc &  number of children &\\\hline
				npm & number of public methods & \\\hline
				rfc & response for a class &number of  methods invoked in response to
				a message to the object.\\\hline
				wmc & weighted methods per class &\\\hline
				nDefects & raw defect counts & Numeric: number of defects found in post-release bug-tracking systems.\\
				defect & defects present? & Boolean: if {\em nDefects} $>0$ then {\em true} else {\em false}
			\end{tabular}
		}
	\end{center}
	\caption{OO measures used in our defect data sets.  Last lines, denote the dependent variables.}\label{fig:ck}
\end{figure*}


\newcommand{\tion}[1]{\S\ref{sect:#1}}
\newcommand{\fig}[1]{Figure~\ref{fig:#1}}
\newcommand{\eq}[1]{Equation~\ref{eq:#1}}
\newcommand{\bi}{\begin{itemize}}
\newcommand{\ei}{\end{itemize}}

\section{Explaining These Results}
To assess the generality of the above results, we need some background knowledge that knows when a few samples will (or will not)
be sufficient to built a defect predictor. Using some sampling theory, this section (a)~builds such a mathematical model; (b)~maps known parameters
of our defect data sets into that model; (c)~identifies what would need to change before the above results would no longer holder.
To start, we repeat the {\em lessons learned} above as well as what is {\em known about defect data sets}.
Next, we define a {\em maths model} which will be used in a {\em Monte Carlo simulation} to generate a log of how many samples are required
to find some signal. This log will be summarized via a {\em decision tree learner}.

\subsection{Set up}
\subsubsection{ Lessons from the above work}

The above results show
that a  
a small number
of examples are sufficient to build a defect predictor, even when the data is being transferred from columns with other names.
In the following we will build a model computing the probability that $n$ training examples are sufficient to detect $e$\% defect modules.

In order to simplify the analysis, we will divide $n$ into
  $n<50, n<100, n<200,n\ge200$ which we denoted {\em lt50, lt100, lt200,ge200}, respectively (and note that {\em ge200} is where the above
  results do not hold).

\subsubsection{Known about defect data sets}\label{sect:data}

Numerous recent
research results show that, for defect data, good
predictors can be built via a {\em median chop} of
numeric project data~\cite{Zhang2014,nam2015clami};
i.e. they are divided into $b=2$ bins. Other results
show that defect prediction data containing dozens
of attributes also contains many correlated
attributes, and that feature subset selection tools
can reduce that set to $d \in \{2,3\}$
dimensions~\cite{Menzies07}.

Hence, in the following, we will pay particular attention to the regions of our mathematical model
that falls into $b=2, d \le 3$.

\subsubsection{A Maths Model}

Before writing down the maths, it is useful to start with some intuitions.
Accordingly, a chess board  containing small  piles of defects in some cells.
Like all chess boards, this one is  divided into a grid of $b^2$ cells (in standard chess, $b=8$ so the board has 64 cells).
Further, some cells of the chessboard are blank which other cells are $e$\% covered
with that signal.

If we throw a small pebble at that chess board, what are the odds
we will hit the defects?
Clearly, this is the product of:
\bi
\item The probability $c$ of picking a particular cell;
\item The probability $p$ that, once we arrive at that cell, we will find  the  signal in that cell.
\ei
With a few changes, this chessboard model can be
used to represent the process of machine
learning. For example,
instead of a board with two
dimensions, data mining works on a ``chessboard''
with $d$ dimensions: one for every independent
variable collected from a project (for example, for a list of the
$d=21$ dimensions used in our MORPH data sets of this study,
see \fig{ck}).

Further,
instead of each dimension being divided into eight (like a chessboard), it is standard practice in data mining for SE~\cite{Menzies2014a}
to divide dimensions accroding to some {\em descritization policy}~\cite{lust08}.
Discretization converts a numeric variable with infinite range into a smaller number of  $b$ bins. Hence, the number of cells in a
hyper-dimensional chessboard is $b^d$ and the probability of selecting any one cell is
\begin{equation}\label{eq:c}c=1/(d^d)=b^{-d}\end{equation}
Once we arrive at any cells, we will be in a region weith $e$ percent errors.
What  is the probability $p$ that we will find those $e$ errors, given $n$ samples from the training data?
According to Voas and Miller~\cite{voas1995software},
if we see something at probability $e$, then we will miss it at probability $1-e$.
After $n$ attempts, the probability of missing it is $(1-e)^n$ so the probability of stumbling onto $e$ errors is
\begin{equation}\label{eq:p}
p(e,n) = (1-(1-e)^n)
\end{equation}
The premise of data mining is that in the data ``chess board'', some cells contain more of the signal than others. Hence, the 
distribution of the $e$ errors are ``skewed'' by some factor $k$. If $k=1$, then all the errors are evenly distributed over all cells.
But at all other values of $k$, some cells contain more errors than others, computed as follows:
    \bi
  \item $R_c$ is a random number $0\le R \le 1$, selected for each part of the space $c\in C$.
  \item $x_c$ is the proportion of errors in each part of $C$. \mbox{$x_c =  R_{c\in C}^k$}.
  \item We normalize $x_c$ to be some ratio $0 \le x_c \le 1$ as follows: $X= \sum_{c\in C} x_c$ then $x_c = x_c/X$
    \ei
    If  $e$ is the ratio of classes within a software projects containing errors, then $e'$ 
    the expected value of the probability $p$ of selecting a cell {\em and} that cell containing errors is:
    \begin{equation}\label{eq:p}
      e' = \sum_{c\in C}cx_ce
    \end{equation}
    where $c$ comes from \eq{c} and $e$ is the ratio of classes in the training set with defects.

Using these equations, we can determine how many training examples $n$ are required before 
  $p(e',n)$ (\eq{p}) returns a value more than some reasonable threshold $T$.
  To make that determination, we call $p(e',n)$ for increasing values of $n$ until $p \ge T$
  (for this paper, we used $T = 67\%$.

\subsubsection{Monte Carlo Simulation}
    This above maths lets us define
    a  Monte Carlo simulation to assess the external validity of our results.
    1000 times, we picked $k,b,d,e$ values at random from:
    \bi
      \item $k \in \{1,2,3,4,5\}$;
  \item $d \in \{3,4,5,6,7\}$ dimensions;
  \item $b \in \{2,3,4,5,6,7\}$ bins; 
    \item $e\in \{0.1,0.2,0.3,0.4\}$
      \ei
      (These ranges were set using our experience with data mining, For example, our prior work shows in defect prediction data sets
      with 40 or more diminesions, that good predictors can be build using $d\le 7$ of those dimensions~\cite{me07a}.)
      
    For each of the 1000 repeats,
     we increased $n$ until \eq{p} reported that $q$ passed our reasonable threshold.
     Next, we logged the 1000 examples of   $k,b,d,e,n$ found via this process.

\subsubsection{Decision Tree Learning}

    This data was given to a decision tree learner to determine what $n$ values are selected by different
    ranges of $\{k,b,d,e\}$. Decision tree learners seek an attribute range that, when used to split the data,
      simplifies the distribution of the dependent variable in each split (in our case, {\em lt50, lt100, lt200,ge200}).
      The decision tree learner is then called recursively on each split.
      To test the stability of the learned model, the learning is repeated ten times, each time using 90\%of the data from training and the rest
      for testing. The weighted average performance values for the learned decision trees were remarkably good:
      \bi
    \item False alarm rates = 2\%;
    \item F-measures (i.e. the harmonic mean of recall and precision) of 95\%
      \ei
\subsection{Results}

The resulting decision tree, shown in \fig{tree}
shows the external validity of these results. In that tree:
\bi
\item
The green branches show where $n \le 200$ samples would
be enough to detect defective modules at probability
67\% or more.
\item
The red branches where more than $n$ samples would be enough.
\ei
When analyzing the tree, we first look at the regions relevant to standard defect data sets (as defined in \tion{data})


The top half of that tree shows that when the data
can be reduced to two dimensions, then if each
dimensions needs less than three bins, then {\em
ltt50, lt100, lt200 } samples will suffice (i.e.$n<
200$).  The bottom of tree shows that when the data
needs more than two or three dimensions, and those
dimensions needs only two bins, then no outcome
needs {\em ge200} samples (i.e. $n\ge200$).

The green branches are interesting since 



% some bracksh shown in white
%leaves need to be labelled A,B,C (little black letter on right hand side)
      
